% Copyright 2015
% 
% Authors
% -------
%   - Emile Contal			emile.contal@cmla.ens-cachan.fr
%   - Cedric Malherbes		cedric.malherbes@cmla.ens-cachan.fr
%   - Thomas Moreau			thomas.moreau@cmla.ens-cachan.fr
%
%
% This file may be distributed and/or modified
%
% 1. under the LaTeX Project Public License and/or
% 2. under the GNU Public License.
%
% See the file doc/licenses/LICENSE for more details.
\pdfminorversion=4
\documentclass[final,mathserif]{beamer}
\mode<presentation> {
  \usetheme{PosterCMLA}
}
\usepackage[size=a0,scale=1.4,debug,orientation=portrait]{beamerposter}

\usepackage{amssymb,amsmath,amsthm,complexity,bm}
\usepackage{tikz,graphicx,subfigure,capt-of,epstopdf,animate,wrapfig,pgfplots}
\usetikzlibrary{arrows,shapes,topaths,calc,patterns}
\graphicspath{{../img/}}
\usepackage[final]{pdfpages}
\usepackage{microtype,nicefrac}
\usepackage{algorithm2e}
\usepackage{pgfplots}
\pgfplotsset{compat=1.8}

\usepackage{shortcuts}

\def\mycite#1{\hfill\textcolor{gray}{#1}}

\newtheorem{prop}{Property}

\setbeamertemplate{itemize item}{\tiny\raise7pt\hbox{\donotcoloroutermaths$\blacktriangleright$}}
\setbeamersize{text margin left=3cm,text margin right=3cm}
\newlength\ColSep
\setlength\ColSep{.5cm}
\newlength\ColWd
\setlength\ColWd{0.5\textwidth-0.66666\ColSep}
\newlength\interCol
\setlength\interCol{0.5em}
\newlength\botBlock
\setlength\botBlock{0.7em}
\newlength\sepBlock
\setlength\sepBlock{-1.5em}

\newcommand{\sepblock}{\hskip1ex\vskip\sepBlock}


% Title information
\title{Global Optimization with Active Learning}
\author{Thomas Moreau\textsuperscript{1}\and Nicolas Vayatis\textsuperscript{1}}
\institute{\textsuperscript{1}CMLA, ENS Cachan, CNRS, Universit\'e Paris Saclay, 94235 Cachan, France}

% Footline with event and location
\event{NIPS 2015}
\location{Montreal, Canada}
\contact{thomas.moreau@ens-cachan.fr}
\date{October 2014}

% Set Logo
\logoLeft{
	\Logo{logo_cachan}{8em}
}
\logoRight{
	\vskip1em
	\mbox{\hskip-8em\Logo{logo_cnrs}{8em}}
	\mbox{\hskip2em\Logo{logo_cmla}{8em}}
}

\begin{document}


\begin{frame}{}
\vspace{0.5ex}

\begin{block}{Abstract}
	We introduce a random search algorithm for non-convex optimization.
	The algorithm is based on the estimation of a confidence set that contains the maximum of the function. The main steps of the estimation procedure are described and numerical experiments are presented to show how the algorithm works practically.\\
  \vspace{1ex}
  {\hspace{1em}\bf Keywords}
  Optimization -- Convolutational Sparse Coding -- Time Series representation
\vskip\botBlock
\end{block}
\sepblock

\begin{block}{1. Motivations and Problem Statement}
	\begin{columns}[t]
\begin{column}{.45\linewidth}
	{\bf Examples of High-dimensional Optimization Problems}
        \begin{itemize}
	\item Optimization in Air Traffic Management
	\item Risk Minimization in Machine Learning
        \item Optimization of Energy Networks
        \end{itemize}

\vskip1em	

{\hspace{0.1em}\bf Derivative-free Optimization} \mycite{\cite{cite101}}\\
        Let $f^{\star}:\X\to\R$ where $\X\subset\R^d$ is a closed set

\begin{itemize}
\item Estimating $\arg\max_{x \in \X} f^{\star}(x) =  \{x \in \X: \forall x' \in \X, f^{\star}(x') \leq f^{\star}(x) \} \,$
\item  Sequential queries $(x_1, f^{\star}(x_1)), (x_2, f^{\star}(x_2)), \dots$
\item Choosing $x_{n+1}$ using the previous observations $\D_n=\{(x_i, f^{\star}(x_i) )\}_{i \leq n}$
\end{itemize}
\end{column}


\begin{column}{.45\linewidth}
	{\hspace{0.1em}\bf Main Objectives}\\
	\begin{itemize}
	\item Design a non-greedy algorithm for non-convex optimization
	\item Keep the computational cost low for large dimensions (d $\gg$ 10)
	\item Confirm the performance empirically
	\item Provide theoretical guarantees
	\end{itemize}
	 

{\hspace{1em}\bf }\\
{\hspace{1em}\bf }\\
{\hspace{1em}\bf }\\
\vspace{1em}

{\hspace{0.1em}\bf Hypothesis and Assumptions}\\
\begin{itemize}
\item The unkown function $f^{\star}$ is continuous
\item The unkown function has no flat parts
\item The unkown function has reasonable variations
\end{itemize}

          %Since $h_{0}^{\star}(x)= \indic{x \in \arg\max_{x \in \X}f(x)}$, 
          
\end{column}
	\end{columns}
\vskip\botBlock
\end{block}

\sepblock




\ColWd=.47\linewidth

\begin{columns}[T]
{\hspace{0.34em}}
    \begin{column}{\ColWd}%\ColWd
      \begin{block}{2. Controlling the Smoothness}
	
	%{\hspace{0.1em}\bf Binary classification for level sets estimation}\\
	%Given a dataset $\D_n$, we have acces to $n+1$ classification datasets: $\D_n^{j}=\{(x_i, \indic{f(x_i) \geq f(x_{(j)})} \}_{i \leq n}$ where $(j)$ stands for the index of the $j^{\textrm{th}}$ smallest value of $\{ f(x_i)\}_{i \leq n}$. Noting $er_{\D^i_n}=\sum_{i \leq n} \indic{h(x_i) \neq y_i}$, we define a set of candidates for the level $i$,
	%$$
	%V(\D^i_n)=\{ h \in \H: er_{\D^i_n}(h)=0\}.
	%$$

	{\hspace{0.1em}\bf Bounding the variations}\\
	To enforce near-by locations to have close associated values, we assume that $f^{\star}$ belongs to a carefully chosen set of functions $\F$. We consider the following sets of functions:
	\begin{itemize}
	 \item Lipschitzian functions: $ \abs{f(x)-f(x')} \leq k \norm{x-x'}_2$
	 \item Gaussian RKHS balls: $\abs{f(x)-f(x')} \leq M\left(1-\exp{\left( -\frac{\norm{x-x'}^2_2}{2\sigma^2} \right) }\right)^{1/2}$
	\end{itemize}

      \begin{figure}[h]
	\begin{center}$
	  \begin{array}{rcl}
	    \includegraphics[width=100mm, height=57mm]{placeholder} & ~~~~~~~~~~~& \includegraphics[width=100mm, height=57mm]{placeholder} ~~~~
	  \end{array}$
	\end{center}
      \caption{Upper and lower bounds for  lipschizian functions (left) and gaussian RKHS (right)} 
       \end{figure}
\vskip\botBlock
\end{block}
\sepblock 
      

\begin{block}{3. Concepts of Active Learning}
	
{\hspace{0.1em}\bf Active learning for binary classification} \mycite{\cite{cite201}}\\
Let $h^{\star}\in \H$ and $\H_n=\{h \in \H: \sum_{i=1}^n \indic{h(X_i) \neq h^{\star}(X_i)}=0 \}$
\begin{itemize}
  \item $DIS(\H_n)=\{x \in \X: \exists (h,h')\in \H_n^2,~h(x) \neq h'(x) \}$
  \item $SUP(\H_n)=\{x \in \X: \forall h \in \H_n,~h(x)=1 \}$
  \item $INF(\H_n)=\{x \in \X: \forall h \in \H_n,~h(x)=0 \}$
\end{itemize}
{\hspace{0.5em}}
{\vspace{-0.5em}}
{\hspace{1em}}
\begin{figure}[h]
\begin{center}$
\begin{array}{cccccc}
\includegraphics[page=1, width=80mm]{placeholder}& ~~~~~~~ & \includegraphics[page=2, width=80mm]{placeholder} & ~~~~~~~ & \includegraphics[page=3, width=80mm]{placeholder} 
\end{array}$
\end{center}
\caption{Computation of the disagreement area for linear classifiers} 
\end{figure}
\vskip\botBlock
\end{block}
\sepblock



\begin{block}{4. Estimating the Maximum with Active Learning}
	
{\hspace{0.1em}\bf Set of candidates} \\
Given a set of mpoints $\{(X_i, f^{\star}(X_i))\}_{i \leq n}$ and a functional space $\F$, we define the set of functions that are still consistent with the datas:
$$
\F_n=\{f \in \F: \forall i \in \{1 \ldots n \},~f(X_i)=f^{\star}(X_i) \}.
$$
{\hspace{0.1em}\bf Query procedure} \\
To keep the consistency of the optimization procedure, we propose a query policy that consists in selecting points uniformely from:
\begin{center}
$\textsc{ArgMax}(\F_n)=\{x \in \X:  ~\exists f \in \F_n $~s.t.~$x \in \arg\max_{x \in \X	} f(x)\}.$
\end{center}
{\hspace{0.1em}\bf Lipschitzian case} \\
If $\F = $Lip($k$), we have that
\\
$ x \in \textsc{ArgMax}(\F_n) $~i.f.f.~$ \min_{i \leq n} f^{\star}(X_i)+ k \norm{x-X_i}_2 \geq \max_{i \leq n } f^{\star}(X_i).$


\vskip\botBlock
\end{block}
\end{column}
\setlength\sepBlock{-.75em}

{\hspace{1.1em}}



    \begin{column}{\ColWd}
      \begin{alertblock}{}%Title
      \end{alertblock}
      
      
      
      
      
      
  
      \begin{block}{5. Selective Algorithm}
	
          \begin{algorithm}[H]
          \SetInd{0.5em}{1em}
          \DontPrintSemicolon
          \textbf{Input:} $\F$, $\X$, $n$ \\ \textbf{Init:} $t, m \leftarrow 0$, $\F_0 \leftarrow \F$, $X_1,X_2,\ldots \iid \mathcal{U}(\X)$ \;	
          \While{$t <n$}{
	    $m \leftarrow m+1$ \\
	    \If{$X_m \in \textsc{ArgMax}(\F_t) $}{
	      Request $f^{\star}(X_m)$, $t\leftarrow t+1$ \\
	      $\F_{t} \leftarrow \{f \in \F_{t-1} : f(X_m)= f^{\star}(X_m)\}$ \;
	    }{}
          }
	\textbf{Output:} Any $ x \in \textsc{ArgMax}(\F_n)$
        \end{algorithm}
\vskip\botBlock
\end{block}
\sepblock





\begin{block}{6. Illustrations}
\begin{figure}[h]
\begin{center}$
\begin{array}{lcr}
\includegraphics[width=128mm, height=128mm]{placeholder} & ~~~~& \includegraphics[width=128mm, height=128mm]{placeholder} \\
~ & ~ & ~\\
\includegraphics[width=128mm, height=128mm]{placeholder} & ~~~~& \includegraphics[width=128mm, height=128mm]{placeholder}
\end{array}$
\end{center}
\caption{Selective Algortihm on four synthetic functions}
\end{figure}

\vskip\botBlock
\end{block}
\sepblock

\begin{block}{7. Open Questions and Discussion}
\begin{itemize}
\item Theoretical performance of the algorithm
\item Comparison with natural competitors
\item Robustness to noise
\end{itemize}
\vskip\botBlock
\end{block}
\sepblock
	
\begin{block}{References}
	{\tiny \bibliography{../sample_library}}
\vskip\botBlock
\end{block}
	
\end{column}

  \end{columns}  
\end{frame}
\end{document}
