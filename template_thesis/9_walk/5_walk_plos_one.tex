\documentclass[../thesis.tex]{subfiles}
\crossref{}
\begin{document}



\section[Rating the Limp in Lower Limb Osteoarthritis]{%
	An Automated Recording Method in Clinical Consultation to Rate the Limp in Lower Limb Osteoarthritis}
	\label{sec:walk:plos_one}


% TL;DR
This section quickly describes a study performed using our signal basis to rate the Limp in Lower Limb Osteoarthritis. The full study can be found in \autoref{chap:walk:osteo}.


% Abstract
For diagnosis and follow up, it is important to be able to quantify limp in an objective, and precise way adapted to daily clinical consultation. The purpose of this exploratory study was to determine if an inertial sensor-based method could provide simple features that correlate with the severity of lower limb osteoarthritis evaluated by the WOMAC index without the use of step detection in the signal processing. Forty-eight patients with lower limb osteoarthritis formed two severity groups separated by the median of the WOMAC index (G1, G2). Twelve asymptomatic age-matched control subjects formed the control group (G0). Subjects were asked to walk straight 10 meters forward and 10 meters back at self-selected walking speeds with inertial measurement units (IMU) (3-D accelerometers, 3-D gyroscopes and 3-D magnetometers) attached on the head, the lower back (L3-L4) and both feet. Sixty parameters corresponding to the mean and the root mean square (RMS) of the recorded signals on the various sensors (head, lower back and feet), in the various axes, in the various frames were computed. Parameters were defined as discriminating when they showed statistical differences between the three groups. In total, four parameters were found discriminating: mean and RMS of the norm of the acceleration in the horizontal plane for contralateral and ipsilateral foot in the doctor’s office frame. No discriminating parameter was found on the head or the lower back. No discriminating parameter was found in the sensor linked frames. This study showed that two IMUs placed on both feet and a step detection free signal processing method could be an objective and quantitative complement to the clinical examination of the physician in everyday practice. Our method provides new automatically computed parameters that could be used for the comprehension of lower limb osteoarthritis. It may not only be used in medical consultation to score patients but also to monitor the evolution of their clinical syndrome during and after rehabilitation. Finally, it paves the way for the quantification of gait in other fields such as neurology and for monitoring the gait at a patient’s home.

% section time_series (end)



\biblio{}
\end{document}