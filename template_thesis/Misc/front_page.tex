\documentclass[../thesis]{subfiles}
\crossref{}
\begin{document}

\clearpage
\newgeometry{top=0.5cm, left=2cm, right=2cm, bottom=1cm}
\thispagestyle{empty}
\vspace{-2cm}

\voffset-10pt


%%%
%% premiere ligne avec logo légaux et numéro national de thèse
%% (modifier les symboles % en début de lignes 
%%  pour obtenir la configuration souhaitée)
%%%
\noindent
\hspace*{-1cm}\hbox{\includegraphics[width=8.6cm]{ed_edmh-h.jpg}} 
\hfill 
\hspace*{-0.4cm}{\parbox[b][4em][c]{12em}{\small \textbf{NNT : 2017SACLN054}}}
%%%
%% Numéro National de Thèse, à préciser lors du second dépôt 
%% de la thèse post-soutenance
%%%
\hfill
%%%
%% logo Etablissement inscripteur
%% (modifier les symboles % en début de lignes 
%% pour obtenir la configuration souhaitée, et ajuster 
%% éventuellement la taille) En cas de cotutelle internationale, 
%% mettre à la place le logo de l'université étrangère et rajouter 
%% en bas de page le logo de l'établissement inscripteur
%%%
\hbox{\includegraphics[width=3.5cm]{logo_ens}}
\vspace{7mm}

\setlength{\fboxrule}{1.5pt}
\fcolorbox{darkred}{white}{\parbox[t][22cm][l]{\textwidth}{
\begin{center}
{\Large\textbf{THÈSE DE DOCTORAT}}
\end{center}
\begin{center}
{de }
\end{center}
\begin{center}
\textsc{\Large l'Université Paris-Saclay}\\
  \vspace*{0.4cm}
École doctorale de mathématiques Hadamard (EDMH, ED 574)\\  
 \vspace*{0.4cm} 
%%%
%% nom établissement inscripteur
%%%
\textit{\small Établissement d'inscription : }
    Ecole normale supérieure de Paris-Saclay\\
 \vspace*{0.2cm} 
%%%
%% nom établissement(s) d'accueil, si différent du précédent
%%%
%\textit{\small établissement d'accueil : }    
%    AgroParisTech\\
%    Commissariat \`a l'énergie atomique et aux énergies alternatives\\
%    Institut des hautes études scientifiques\\
%    Institut national de la recherche agronomique\\
%\vspace*{0.2cm} 
%%%
%% nom laboratoire(s) d'accueil
%%%
\textit{\small Laboratoire d'accueil : }
  Centre de mathématiques et de leurs applications, UMR 8536 CNRS\\
\vspace*{0.2cm}
\end{center}




\begin{center}
\textit{Spécialité de doctorat : } 
%%%
%%  Choisir l'une des trois spécialités suivantes (soumis à
%%  accord du comité de direction de l'EDMH)
%%%
%{\large Mathématiques fondamentales}
{\large Mathématiques appliquées}
%{\large Mathématiques aux interfaces}
\end{center}

\vspace{5mm}

\begin{center}
{\large\textbf{Thomas MOREAU}}
\end{center}

\vspace{3mm}
 
\begin{center}
{\Large \TITLE{}\\[4em]}
%%%
%% éventuellement sur deux lignes
%%%
\end{center}


\vspace{10mm}

\noindent{\small \textit{Date de soutenance~: }} 19 Décembre 2017

\vspace{5mm}

\noindent
\textit{\small Après avis des rapporteurs~: }
\begin{tabular}{l}
\textsc{Julien MAIRAL} (INRIA Grenoble)\\
\textsc{Stéphane MALLAT} (École Normal Supérieure)\\
\textsc{René VIDAL} (Université Johns Hopkins)\\
\end{tabular}

\vspace{8mm}

\noindent
\textit{\small Jury de soutenance~: }
%%%
%% par ordre alphabétique des membres
%%
%% version avant soutenance à adapter (un jury peut contenir 
%% tout ou partie des rapporteurs, tout au partie des codirecteurs de thèses,
%% voire des invités). Le président du jury est choisi par le jury en 
%% son sein le jour de la soutenance, et est indiqué sur les exemplaires 
%% post-soutenance
%% En cas d'absense d'un membre de jury prévu, les exemplaires post-thèse 
%% ne doivent faire figurer que les présents.
%%%%
\begin{tabular}{ll}
\textsc{Stéphanie ALLASSONNIÈRE}&(Université Paris-Descartes) {\small Présidente}\vspace{1mm}\\
\textsc{Alexandre GRAMFORT}&(INRIA Saclay) {\small Examinateur}\vspace{1mm}\\
\textsc{Julien MAIRAL}&(INRIA Grenoble) {\small Rapporteur}\vspace{1mm}\\
\textsc{Stéphane MALLAT}&(École Normale Supérieure) {\small Rapporteur}\vspace{1mm}\\
\textsc{Laurent OUDRE}&(Université Paris 13) {\small Codirecteur de thèse}\vspace{1mm}\\
\textsc{Nicolas VAYATIS}&(ENS Paris-Saclay) {\small Codirecteur de thèse}\vspace{1mm}\\
\textsc{Pierre-Paul VIDAL}&(Université Paris-Descartes) {\small Examinateur}\vspace{1mm}\\
\textsc{René VIDAL}&(Université Johns Hopkins) {\small Rapporteur}\vspace{1mm}\\
\end{tabular}
}}

\vfill
\noindent
\hbox{\includegraphics[width=2.2cm]{logo_fmjh.jpg}}
%\hfill 
%%%
%% logo Etablissement d'accueil, si différent du précédent
%% (modifier les symboles % en début de lignes 
%% pour obtenir la configuration souhaitée, et ajuster 
%% éventuellement la taille)
%%%
%\hbox{\includegraphics[width=2cm]{logoIHES.jpg}}
%\hbox{\includegraphics[width=2.1cm]{logoINRA.jpg}}
%\hbox{\includegraphics[width=2.5cm]{logoAgro.jpg}}
%\hbox{\includegraphics[width=1.5cm]{logoCEA.jpg}}
\hfill
%%%
%% logo laboratoire(s) d'accueil, s'il existe
%%%
\hbox{\includegraphics[width=2.4cm]{logo_cmla}}
\hfill \includegraphics[width=1cm]{pictoParis-Saclay.jpg}

\end{document}

