\documentclass[../thesis.tex]{subfiles}
\crossref{}
\begin{document}


	In this part, we focus on pattern-based representations. \autoref{part1}
	is organized as follows. \autoref{chap:csc} introduces the convolutional representation
	model as well as state-of-the-art algorithms used to extract patterns from signals. Then,
	\autoref{chap:ssa} introduces the Singular Spectrum Analysis (SSA) and shows that it
	actually corresponds to a convolutional representation with specific patterns. Then, a
	framework to automatically improve the interpretability of the components computed with
	the SSA is evaluated. Finally, \autoref{chap:dicod} presents a novel algorithm based on
	greedy coordinate descent to solve the convolutional sparse coding. This algorithm can
	be distributed asynchronously to represent long signals in the convolutional representation
	model. It is proven to converge and to have a super-linear speedup compared to the
	classical greedy coordinate descent algorithm.

\biblio{}
\end{document}