\documentclass[landscape,a0paper,fontscale=0.265]{baposter} %fontscale=0.277

\usepackage{calc}
%\usepackage{fourier}
\newcommand\Warning{%
 \makebox[1.4em][c]{%
 \makebox[0pt][c]{\raisebox{.1em}{\small!}}%
 \makebox[0pt][c]{\color{red}\Large$\bigtriangleup$}}}
\usepackage{graphicx}
\usepackage{amsmath}
\usepackage{amssymb}
\usepackage{relsize}
\usepackage{multirow}
\usepackage{rotating}
\usepackage{bm}
\usepackage{url}
\usepackage{cancel}
\usepackage{array}

\usepackage{graphicx}
\usepackage{xargs}
\usepackage{multicol}

\usepackage{enumitem}

%\usepackage{times}
%\usepackage{helvet}
%\usepackage{bookman}
\usepackage{palatino}
\usepackage{subfiles}


\usepackage{ragged2e}
\usepackage{setspace}

\usepackage{anyfontsize}



\usepackage[pages=some]{background}

\newcommand{\captionfont}{\footnotesize}



\graphicspath{{images/}{../img/}}

\usepackage{empheq}
\usepackage{dsfont}
\usepackage{enumitem}
\usepackage{stmaryrd}

%%%%%%%% added by mainak %%%%%%%%%%%
\usepackage{mathsymbols}
\usepackage[scaled]{helvet}
%\usepackage{nimbussans}
\usepackage[T1]{fontenc}
\usepackage{subfigure}
\usepackage{bbm}

% Import tikz packages for subfiles
\usepackage{tikz}
\usetikzlibrary{arrows, positioning, calc}
\usetikzlibrary{decorations.pathreplacing, patterns}

\usepackage[ruled]{algorithm2e}
% change font size of algorithm
\SetAlFnt{\small}
\SetAlCapFnt{\small}
\SetAlCapNameFnt{\small}
\setlength{\algomargin}{0em}
\SetAlCapHSkip{0em}
\usepackage{algorithmic}

%%%%%%%%%%%%%%%%%%%%%%%%%%%%%%%%%%%%

% \newtheorem{thm}{Theorem}

%%%%%%%%%%%%%%%%%%%%%%%%%%%%%%%%%%%%%%%%%%%%%%%%%%%%%%%%%%%%%%%%%%%%%%%%%%%%%%%%
%%%% Some math symbols used in the text
%%%%%%%%%%%%%%%%%%%%%%%%%%%%%%%%%%%%%%%%%%%%%%%%%%%%%%%%%%%%%%%%%%%%%%%%%%%%%%%%

%%%%%%%%%%%%%%%%%%%%%%%%%%%%%%%%%%%%%%%%%%%%%%%%%%%%%%%%%%%%%%%%%%%%%%%%%%%%%%%%
% Multicol Settings
%%%%%%%%%%%%%%%%%%%%%%%%%%%%%%%%%%%%%%%%%%%%%%%%%%%%%%%%%%%%%%%%%%%%%%%%%%%%%%%%
\setlength{\columnsep}{1.5em}
\setlength{\columnseprule}{0mm}

%%%%%%%%%%%%%%%%%%%%%%%%%%%%%%%%%%%%%%%%%%%%%%%%%%%%%%%%%%%%%%%%%%%%%%%%%%%%%%%%
% Save space in lists. Use this after the opening of the list
%%%%%%%%%%%%%%%%%%%%%%%%%%%%%%%%%%%%%%%%%%%%%%%%%%%%%%%%%%%%%%%%%%%%%%%%%%%%%%%%
\newcommand{\compresslist}{%
\setlength{\itemsep}{1pt}%
\setlength{\parskip}{0pt}%
\setlength{\parsep}{0pt}%
}

%%%%%%%%%%%%%%%%%%%%%%%%%%%%%%%%%%%%%%%%%%%%%%%%%%%%%%%%%%%%%%%%%%%%%%%%%%%%%%
%%% Begin of Document
%%%%%%%%%%%%%%%%%%%%%%%%%%%%%%%%%%%%%%%%%%%%%%%%%%%%%%%%%%%%%%%%%%%%%%%%%%%%%%


\begin{document}

%%%%%%%%%%%%%%%%%%%%%%%%%%%%%%%%%%%%%%%%%%%%%%%%%%%%%%%%%%%%%%%%%%%%%%%%%%%%%%
%%% Here starts the poster
%%%---------------------------------------------------------------------------
%%% Format it to your taste with the options
%%%%%%%%%%%%%%%%%%%%%%%%%%%%%%%%%%%%%%%%%%%%%%%%%%%%%%%%%%%%%%%%%%%%%%%%%%%%%%
% Define some colors

%\definecolor{lightblue}{cmyk}{0.83,0.24,0,0.12}
\definecolor{lightblue}{rgb}{0.145,0.6666,1}
\definecolor{tpt}{RGB}{200,34,84}
\definecolor{goldengatered}{RGB}{255,41,46}
% \definecolor{orange}{RGB}{31,58,224}
%\definecolor{inriared}{HTML}{E52D37}

\def\maincolor{lightblue}

\newcommand{\mydot}{\hspace*{-0.3ex}%
\raisebox{0.2ex}{\color{\maincolor}\rule{1.1ex}{1.1ex}}%
\hspace*{0.4ex}%
}

\hyphenation{resolution occlusions}
%%
\begin{poster}%
  % Poster Options
  {
  % Show grid to help with alignment
  grid=false,
  columns=3,
  % Column spacing
  colspacing=0.35em,
  % Color style
  bgColorOne=white,
  bgColorTwo=lightblue,
  % borderColor=goldengatered,
  borderColor=black,
  % headerColorOne=goldengatered,
  headerColorOne=\maincolor,
  headerColorTwo=purple,
  headerFontColor=white,
  boxColorOne=white,
  boxColorTwo=lightblue,
  % Format of textbox
  textborder=rectangle,
  % Format of text header
  eyecatcher=true,
  headerborder=closed,
  headerheight=0.148\textheight,
%  textfont=\sc, An example of changing the text font
  headershape=rectangle,
  headershade=plain,
  headerfont=\Large\fontfamily{\sfdefault}\bfseries, %Sans Serif
  textfont={\setlength{\parindent}{1.5em}},
  boxshade=plain,
 % background=shadelr,
  background=plain,
  linewidth=0.5pt
  }
  % Eye Catcher
  {
  % \rotatebox[origin=c]{-90}{\includegraphics[height=16em]{images/logo_inria.pdf}}
  \begin{minipage}{.17\linewidth}
    \centering
    \includegraphics[width=0.6\linewidth]{logo_cmla}
  \end{minipage}
  }
  % Title
  {\huge\fontfamily{\sfdefault}\bfseries {
    Learning step sizes for unfolded sparse coding
  }
    \vspace{.7em}
  }
  % Authors
  {\Large  {\textbf
    {
      Pierre Ablin$^{*}$, \hspace{5pt}
      Thomas Moreau$^{*}$, \hspace{5pt}
      Mathurin Massias, \hspace{5pt}
      Alexandre Gramfort}}\\
       \vspace{5pt}
   \normalsize{
   Univ. Paris-Saclay, INRIA, Parietal team, Saclay, France.
   $^*$ Contributed Equally
   \\}
   \vspace{-10pt}
  }
  % University logo
  {% The makebox allows the title to flow into the logo, this is a hack because of the L shaped logo.
    %\includegraphics[height=7.5em]{images/logo_sb.pdf}
    %\includegraphics[height=3.0em]{images/bilgi.jpg}
    \begin{minipage}{.18\linewidth}
     \hfill
      \begin{minipage}{\linewidth}
        \centering
        \includegraphics[width=0.9\linewidth]{example-image-golden}\\
        \includegraphics[width=.8\linewidth]{logo_cachan}\\
      \end{minipage}
    \end{minipage}
  }

%%%%%%%%%%%%%%%%%%%%%%%%%%%%%%%%%%%%%%%%%%%%%%%%%%%%%%%%%%%%%%%%%%%%%%%%%%%%%%
%%% Now define the boxes that make up the poster
%%%---------------------------------------------------------------------------
%%% Each box has a name and can be placed absolutely or relatively.
%%% The only inconvenience is that you can only specify a relative position
%%% towards an already declared box. So if you have a box attached to the
%%% bottom, one to the top and a third one which should be in between, you
%%% have to specify the top and bottom boxes before you specify the middle
%%% box.
%%%%%%%%%%%%%%%%%%%%%%%%%%%%%%%%%%%%%%%%%%%%%%%%%%%%%%%%%%%%%%%%%%%%%%%%%%%%%%

% A coloured circle useful as a bullet with an adjustably strong filling
\newcommand{\colouredcircle}{%
  \tikz{\useasboundingbox (-0.2em,-0.32em) rectangle(0.2em,0.32em);
        \draw[draw=black,fill=purple,line width=0.03em] (0,0) circle(0.18em);}}


%%%%%%%%%%%%%%%%%%%%%%%%%%%%%%%%%%%%%%%%%%%%%%%%%%%%%%%%%%%%%%%%%%%%%%%%%%%%%%
\headerbox{Solving sparse linear inverse problems}{name=intro,column=0,row=0}{
%%%%%%%%%%%%%%%%%%%%%%%%%%%%%%%%%%%%%%%%%%%%%%%%%%%%%%%%%%%%%%%%%%%%%%%%%%%%%%
\begin{itemize}[leftmargin=*,nolistsep]
\setlength\itemsep{0.01em}
\item \textbf{Objective:} Solve the Lasso for a fixed $D$ and $x \sim \mathbb P$.
\[
  z^*(x) = \argmin_z F_x(z) = \frac{1}{2}\|x - Dz\|_2^2 + \lambda \|z\|_1
\]%
\vskip-1em
\item \textbf{Iterative algorithm:} Use ISTA for each $x$
\item \textbf{Deep Learning:} Use a NN to learn a mapping
\[
  \Phi^T_\Theta: x \mapsto z^*(x); \quad \text{ for } x \sim \mathbb P
\]
\end{itemize}
}



%%%%%%%%%%%%%%%%%%%%%%%%%%%%%%%%%%%%%%%%%%%%%%%%%%%%%%%%%%%%%%%%%%%%%%%%%%%%%%
\headerbox{Learned ISTA \hskip11.5ex\normalsize {[Gregor \& Le Cun 2010]}}{name=lista,column=0,below=intro,span=1}{
%%%%%%%%%%%%%%%%%%%%%%%%%%%%%%%%%%%%%%%%%%%%%%%%%%%%%%%%%%%%%%%%%%%%%%%%%%%%%%
\raggedright
\textbf{ISTA:}
$
  z^{(t+1)} = \text{ST}(z^{(t)} - {\color{orange}\gamma} D^\top(Dz^{(t)} - x), {\color{orange}\gamma}\lambda)
$,\\
where ${\color{orange}\gamma}$ is the step size, usually chosen as ${\color{orange}1/L}$.\\[.5em]
\textbf{LISTA:} Let $W_z = I_m  - {\color{orange}\gamma} D^\top D;~W_x = {\color{orange}\gamma} D^\top$ and ${\color{orange}\beta}={\color{orange}\gamma}$\\[1em]

\begin{minipage}{.55\textwidth}
  $z^{(t+1)} = \text{ST}(W_z z^{(t)} + W_x x, \lambda {\color{orange}\beta})$\\[.5em]
  \textbf{Re-parametrization:}\\[.5em]
    $W_z = I_m - {\color{orange}\alpha} W^\top D;~ W_x = {\color{orange}\alpha} W^\top$
\end{minipage}%
\begin{minipage}{.45\textwidth}
    \centering
    skjhfkjhdskjfdfh
\end{minipage}

\vskip1.5em
\mydot Learn parameters $\Theta = \{W^{(t)}, {\color{orange}\alpha}^{(t)}, {\color{orange}\beta}^{(t)}\}$
\vskip1em
\begin{minipage}{.33\textwidth}
  \centering
  \textbf{Supervised learning}\\[.5em]
  Ground truth available $s_1, \dots, s_N$
  \[
    \sum_{i=1}^N (\Phi^T_\Theta(x_i) - s_i)^2
  \]
\end{minipage}\vline
\begin{minipage}{.33\textwidth}
  \centering
  \textbf{Semi-supervised learning}\\[.5em]
  Compute $s_i = \argmin_z F_{x_i}(z)$
  \[
    \sum_{i=1}^N (\Phi^T_\Theta(x_i) - s_i)^2
  \]
\end{minipage}\vline
\begin{minipage}{.33\textwidth}
  \centering
  \textbf{Unsupervised learning}\\[.5em]
  Learn to solve the Lasso
  \[
    \sum_{i=1}^N F_{x_i}(\Phi^T_\Theta(x_i))
  \]
\end{minipage}%

}


%%%%%%%%%%%%%%%%%%%%%%%%%%%%%%%%%%%%%%%%%%%%%%%%%%%%%%%%%%%%%%%%%%%%%%%%%%%%%%
\headerbox{Local smoothness constants}{name=theory,column=0,below=lista, span=1}{
%%%%%%%%%%%%%%%%%%%%%%%%%%%%%%%%%%%%%%%%%%%%%%%%%%%%%%%%%%%%%%%%%%%%%%%%%%%%%%
\noindent
\mydot ${\color{orange}L} = \max \|Dz\|_2^2$ subject to $\|z\|_2=1$
{\small
\begin{align*}
  F_x(z) & = f_x(z^{(t)}) + \langle \nabla f_x(z^{(t)}), z - z^{(t)}\rangle + \frac{1}{2}\|D(z-z^{(t)})\|_2^2 + \lambda\|z\|_1\\
   & \le f_x(z^{(t)}) + \langle \nabla f_x(z^{(t)}), z - z^{(t)}\rangle + \frac{{\color{orange}L}}{2}\|z-z^{(t)}\|_2^2 + \lambda\|z\|_1
\end{align*}
}
\mydot ${\color{orange}L_S} = \max \|Dz\|_2^2$ subject to $\|z\|_2=1, Supp(z) \subset S$.\\[.6em]
{\centering ISTA with {\bf greater step-size}: {\color{orange}$\gamma = 1/L_s$}\\}
}



%%%%%%%%%%%%%%%%%%%%%%%%%%%%%%%%%%%%%%%%%%%%%%%%%%%%%%%%%%%%%%%%%%%%%
%%%%%%%%%%%%%%%%%%%%%%%%%%%%%%%%%%%%%%%%%%%%%%%%%%%%%%%%%%%%%%%%%%%%%
%%%%%%%%%%%%%%%%%%%%%%%%%%%%%%%%%%%%%%%%%%%%%%%%%%%%%%%%%%%%%%%%%%%%%
%%%%%%%%%%%%%%%%%%%%%%%%%%%%%%%%%%%%%%%%%%%%%%%%%%%%%%%%%%%%%%%%%%%%%
%%%%%%%%%%%%%%%%%%%%%%%%%%%%%%%%%%%%%%%%%%%%%%%%%%%%%%%%%%%%%%%%%%%%%
%%%%%%%%%%%%%%%%%%%%%%%%%%%%%%%%%%%%%%%%%%%%%%%%%%%%%%%%%%%%%%%%%%%%%
%%%%%%%%%%%%%%%%%%%%%%%%%%%%%%%%%%%%%%%%%%%%%%%%%%%%%%%%%%%%%%%%%%%%%
%%%%%%%%%%%%%%%%%%%%%%%%%%%%%%%%%%%%%%%%%%%%%%%%%%%%%%%%%%%%%%%%%%%%%

%%%%%%%%%%%%%%%%%%%%%%%%%%%%%%%%%%%%%%%%%%%%%%%%%%%%%%%%%%%%%%%%%%%%%%%%%%%%%%
\begin{posterbox}[name=center,column=1,row=0,span=1,
                  boxheaderheight=0em, height=1,
                  boxColorOne=\maincolor,]{}
%%%%%%%%%%%%%%%%%%%%%%%%%%%%%%%%%%%%%%%%%%%%%%%%%%%%%%%%%%%%%%%%%%%%%%%%%%%%%%
    \vskip3em
    \noindent
    \LARGE \color{white}\underline{\textbf{Theorem:}} The weights of a neural network trained to solve the lasso asymptotically only learn a step size.
    \[
      \frac{\alpha^{(t)}}{\beta^{(t)}} W^{(t)} \xrightarrow[t\to\infty]{} D
    \]
    \vskip2em
  \centering
    \includegraphics[width=.8\textwidth]{example-image-a}\\[3em]
     { \centering
      \includegraphics[width=.35\textwidth]{QR_unitag}\\[1em]}
\end{posterbox}



%%%%%%%%%%%%%%%%%%%%%%%%%%%%%%%%%%%%%%%%%%%%%%%%%%%%%%%%%%%%%%%%%%%%%%%%%%%%%%
\headerbox{Improving ISTA step-size}{name=exp,column=2,row=0,span=1}{
%%%%%%%%%%%%%%%%%%%%%%%%%%%%%%%%%%%%%%%%%%%%%%%%%%%%%%%%%%%%%%%%%%%%%%%%%%%%%%
    \raggedright

  % \textbf{Real data} \\
\textit{Better step-sizes for ISTA}\\[.4em]
\vskip2pt
\begin{itemize}[nolistsep,leftmargin=*]\itemsep.4em
	\item Back-tracking line-search
  \item {\bf OISTA:} Adapt step-sizes to {\bf Local smoothness constants ${\color{orange}L_S}$}\\
  \item {\bf SLISTA:} Learn only step-sizes with LISTA.
\end{itemize}
  \vskip.5em
  \includegraphics[width=0.5\columnwidth]{example-image-a}%
  \includegraphics[width=0.5\columnwidth]{example-image-a}
  \vskip1em
  {\centering The step-sizes learned by SLISTA tend to be in $[{\color{orange}\frac{1}{L_s}}, {\color{orange}\frac{2}{L_S}}]$.\\[.5em]}
  {\color{white}.}
 }

%%%%%%%%%%%%%%%%%%%%%%%%%%%%%%%%%%%%%%%%%%%%%%%%%%%%%%%%%%%%%%%%%%%%%%%%%%%%%%
\headerbox{Varying the sparsity}{name=next,column=2,below=exp, span=1}{
%%%%%%%%%%%%%%%%%%%%%%%%%%%%%%%%%%%%%%%%%%%%%%%%%%%%%%%%%%%%%%%%%%%%%%%%%%%%%%
\raggedright
SLISTA works better when $z^*$ is sparse as this reduces ${\color{orange}L_S}$.\\[1em]
\includegraphics[width=\linewidth]{example-image-golden}
}



%%%%%%%%%%%%%%%%%%%%%%%%%%%%%%%%%%%%%%%%%%%%%%%%%%%%%%%%%%%%%%%%%%%%%%%%%%%%%%
\headerbox{References}{name=ref,column=2,below=next, span=1}{
%%%%%%%%%%%%%%%%%%%%%%%%%%%%%%%%%%%%%%%%%%%%%%%%%%%%%%%%%%%%%%%%%%%%%%%%%%%%%%
%\vspace{-0.1cm}
\small{
\begin{itemize}[itemsep=.3em,nolistsep,leftmargin=*]
  \item Gregor, K. \& Le Cun, Y. (2010) {\color{lightblue}  Learning Fast Approximations of Sparse Coding.} ICML.
  % \item Giryes, R., Eldar, Y. C., Bronstein, A. M. \& Sapiro, G. (2018) {\color{lightblue} Tradeoffs between Convergence Speed and Reconstruction Accuracy in Inverse Problems.} IEEE Trans. on Sig. Proc.
  \item Chen, X., Liu, J., Wang, Z. \& Yin, W. (2018) {\color{lightblue} Theoretical Linear Convergence of Unfolded ISTA and its Practical Weights and Thresholds.} NeurIPS.
  \item Liu, J., Chen, X., Wang, Z. \& Yin, W (2019). {\color{lightblue} ALISTA: Analytic Weights are as good as Learned weigths in LISTA.} ICLR.
  \item Moreau, T. \& Bruna, J. (2017). {\color{lightblue} Understanding Trainable Sparse Coding with Matrix Factorization.} ICLR.

\end{itemize}
}
%\vspace{-0.1cm}
}


% %%%%%%%%%%%%%%%%%%%%%%%%%%%%%%%%%%%%%%%%%%%%%%%%%%%%%%%%%%%%%%%%%%%%%%%%%%%%%%
% \headerbox{Acknowledgement}{name=ack,column=2,below=ref, span=1}{
% %%%%%%%%%%%%%%%%%%%%%%%%%%%%%%%%%%%%%%%%%%%%%%%%%%%%%%%%%%%%%%%%%%%%%%%%%%%%%%
% \raggedright
% \vspace{2pt}
% \small{
% French National Research Agency grants ANR-14-NEUC-0002-01 and ANR-16-CE23-0014, and the ERC Starting Grant SLAB ERC-YStG-676943
% } \vspace{1pt}
% }

\end{poster}

\end{document}

